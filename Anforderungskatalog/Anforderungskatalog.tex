
% Dieses Dokument ist eine Vorlage zum Erstellen eines Anforderungskatalogs.

% Deklaration der Variablen
\author{Tomas Fürl}
\newcommand{\Projektname}{Vorlagenprojekt}
\title{Anforderungen für \Projektname}
\documentclass[10pt,a4paper]{article}
\usepackage[utf8]{inputenc}
\usepackage[german]{babel}
\usepackage[german]{isodate}

% Ändere Paragraph zu Überschrift 4
\usepackage{titlesec}
\setcounter{secnumdepth}{4}
\titleformat{\paragraph}
{\normalfont\normalsize\bfseries}{\theparagraph}{1em}{}
\titlespacing*{\paragraph}
{0pt}{3.25ex plus 1ex minus .2ex}{1.5ex plus .2ex}

\usepackage[scaled]{helvet}
\renewcommand\familydefault{\sfdefault}

\begin{document}

% Titel, Inhaltsverzeichnis, Bildverzeichnis, Tabellenverzeichnis
    \pagenumbering{gobble}
    \maketitle
    \newpage
    \pagenumbering{Roman}
    \tableofcontents
    % \newpage
    % \listoffigures
    % \newpage
    % \listoftables
    \newpage
    \pagenumbering{arabic}

    \section{Funktionale Anforderungen}
    \subsection{Kategorie}
    \subsubsection{\textbf{Funktionstitel}}
    \paragraph{Kurzbeschreibung}
    \paragraph{Aktor}
    \paragraph{Vorbedingungen}
    \paragraph{Beschreibung des Ablaufs}
    \paragraph{Auswirkungen}
    \paragraph{Anmerkungen}

    \section{Nicht-Funktionale Anforderungen}
    \section{Listenbeispiel}
    \begin{itemize}
        \item One entry in the list
        \item Another entry in the list
    \end{itemize}
    \begin{enumerate}
        \item One entry in the list
        \item Another entry in the list
    \end{enumerate}
    \section{Vorlage Anforderung}
    \subsection{Kategorie}
    \subsubsection{\textbf{Funktionstitel}}
    \paragraph{Kurzbeschreibung}
    \paragraph{Aktor}
    \paragraph{Vorbedingungen}
    \paragraph{Beschreibung des Ablaufs}
    \paragraph{Auswirkungen}
    \paragraph{Anmerkungen}

\end{document}