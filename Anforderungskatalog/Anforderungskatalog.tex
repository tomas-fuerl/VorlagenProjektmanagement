
% Dieses Dokument ist eine Vorlage zum Erstellen eines Anforderungskatalogs.

% Deklaration der Variablen
\author{Tomas Fürl}
\newcommand{\Projektname}{Vorlagenprojekt}
\title{Anforderungen für \Projektname}
\documentclass[10pt,a4paper]{article}
\usepackage[utf8]{inputenc}
\usepackage[german]{babel}
\usepackage[german]{isodate}
\usepackage{tabularx}

% Ändere Paragraph zu Überschrift 4
\usepackage{titlesec}
\setcounter{secnumdepth}{4}
\titleformat{\paragraph}
{\normalfont\normalsize\bfseries}{\theparagraph}{1em}{}
\titlespacing*{\paragraph}
{0pt}{3.25ex plus 1ex minus .2ex}{1.5ex plus .2ex}

\usepackage[scaled]{helvet}
\renewcommand\familydefault{\sfdefault}

\begin{document}

% Titel, Inhaltsverzeichnis, Bildverzeichnis, Tabellenverzeichnis
    \pagenumbering{gobble}
    \maketitle
    \newpage
    \pagenumbering{Roman}
    \tableofcontents
    % \newpage
    % \listoffigures
    % \newpage
    % \listoftables
    \newpage
    \pagenumbering{arabic}

    \section{Funktionale Anforderungen}

    \section{Nicht-Funktionale Anforderungen}

    \section{Noch nicht vollständig oder sortiert}

    % Vorlage für Anforderung
    \section{Vorlage für eine Anforderung}
    \subsection{Anforderungs-Kategorie}
    \textbf{\subsubsection{Anforderungstitel}}
    \paragraph{Kurzbeschreibung}
    Hier wird kurz die Anforderung beschrieben.
    \paragraph{Aktor}
    Wer ist beteiligt?
    \paragraph{Vorbedingungen}
    Was muss vorher gemacht werden / gemacht sein oder existieren, 
    dass diese Anforderung sachgemäß abläuft.
    \paragraph{Beschreibung des Ablaufs}
    Hier werden die Ablaufschritte deutlich gemacht. Das kann auch 
    in Form eines BPMN-Diagrammes oder Flussdiagamms geschehen.
    \paragraph{Auswirkungen}
    Was ist das Endprodukt, welche Funktionalität wird danach 
    angesprochen, wer ist davon betroffen, wann und wo?
    \paragraph{Anmerkungen}
    Hier sollen weitere Anmerkungen eingetragen werden.

\end{document}