% Dieses Dokument ist eine Vorlage zum Erstellen eines Lastenheftes.
% Definition lt. Wikipedia:
% Das Lastenheft [...] beschreibt die Gesamtheit der Anforderungen des Auftraggebers an die 
% Lieferungen und Leistungen eines Auftragnehmers.

% Dokument im Artkelformat auf A4 mit Grundgröße 12pt
\documentclass[a4paper, 12pt]{article}

% Deutsche Formate der Funktionen
\usepackage[ngerman]{babel}

% Helvetica als Schriftart
\usepackage[scaled]{helvet}
\usepackage[T1]{fontenc}

% Module nachladen
\usepackage{tabu}

% Deklaration der Variablen
\newcommand{\Projektname}{Vorlagenprojekt}
\title{Lastenheft \Projektname}
\date{Erstellt am: \today}
\author{Tomas Fürl}

\renewcommand\familydefault{\sfdefault}
\begin{document}

% Titelseite
    \pagenumbering{gobble}
    \maketitle
    \newpage
    \pagenumbering{arabic}

% Projektbeschreibung
    \section{Projektbeschreibung}
    \subsection{Ziele des Projekts}
    Kurz und prägnant Arbeitstitel und Ziel des Projekts in 1 - 2 beschreiben.
    \subsection{Terminierung}
    Gewünschte Ablauf- und Terminplanung
    \subsection{Kosten}
    Vorstellung über die Kosten

% Ist-Zustand
    \section{Ist-Zustand}
    Hier ist die aktuelle Situation darzustellen:
    \begin{itemize}
        \item Was und wie viel verarbeitet das aktuelle Anwendungssystem?
        \item Wer benutzt das Softwaresystem?
        \item Wie wird das Softwaresystem organisiert?
    \end{itemize}
    Hier können noch Grafiken zur Unternehmensstruktur, Prozessen, ... beigefügt werden.

\end{document}