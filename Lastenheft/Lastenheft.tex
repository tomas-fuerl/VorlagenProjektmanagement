% Dieses Dokument ist eine Vorlage zum Erstellen eines Lastenheftes.
% Definition lt. Wikipedia:
% Das Lastenheft [...] beschreibt die Gesamtheit der Anforderungen des Auftraggebers an die 
% Lieferungen und Leistungen eines Auftragnehmers.

% Deklaration der Variablen
\author{Tomas Fürl}
\title{Vorlagenprojekt}
\newcommand{\Projektname}{Vorlagenprojekt}
\documentclass[10pt,a4paper]{article}
\usepackage[utf8]{inputenc}
\usepackage[german]{babel}
\usepackage[german]{isodate}
\usepackage[scaled]{helvet}
\renewcommand\familydefault{\sfdefault}

\begin{document}

% Titel, Inhaltsverzeichnis, Bildverzeichnis, Tabellenverzeichnis
\pagenumbering{gobble}
\maketitle
\newpage
\pagenumbering{Roman}
\tableofcontents
\newpage
\listoffigures
\newpage
\listoftables
\newpage
\pagenumbering{arabic}

% Projektbeschreibung
    \section{Projektbeschreibung}
    \subsection{Ziele des Projekts}
    Kurz und prägnant Arbeitstitel und Ziel des Projekts in 1 - 2 beschreiben.
    \subsection{Terminierung}
    Gewünschte Ablauf- und Terminplanung
    \subsection{Kosten}
    Vorstellung über die Kosten

% Ist-Zustand
    \section{Ist-Zustand}
    Hier ist die aktuelle Situation darzustellen:
    \begin{itemize}
        \item Was und wie viel verarbeitet das aktuelle Anwendungssystem?
        \item Wer benutzt das Softwaresystem?
        \item Wie wird das Softwaresystem organisiert?
    \end{itemize}
    Hier können noch Grafiken zur Unternehmensstruktur, Prozessen, ... beigefügt werden.

% Soll-Zustand
    \section{Soll-Zustand}
    Hier soll beschrieben werden, wie der Auftraggeber sich das neue System vorstellt und 
    welche anforderungen er an das System stellt.
    Hierbei muss eine Liste der Anforderungen erstellt werden.
    Diese müssen nach MUSS, SOLL und KANN kategorisert werden.
    das bietet sich Tabellarisch an.

% Schnittstellen
    \section{Schnittstellen}
    Hier Sollen die Schnittstellen genannt werden, die das Projekt später betreffen werden, 
    z.B. Tabellen, die importiert werden müssen, andere Programme, in denen die Software 
    integriert wird, und APIs, die später erfüllt werden müssen werden.
    Welche Kommunikatiionsprotokolle sollen verwendet werden?
    Wie wird die Software ausgerollt?

% Inbetriebnahme und Einsatz
    \section{Inbetriebnahme und Einsatz}
    \subsection{Dokumentation}
    In welchem Format, Umfang, für wen und in welchen Zeitabständen bei welchen Zeitpunkten von wem 
    soll Dokumentation erstellt werden.
    \subsection{Testing}
    In welcher Art und Weise, von wem, wann sollen die verschiedenen Bestandteile getestet werden sollen.
    Außerdem werden Gewährleistungskriterien festgelegt.
    \subsection{Schulungen}
    Wer führt wo, wann, wie Mitarbeiterschulungen durch.
% Qualitätsanforderungen
    \section{Qualitätsanforderungen}
    Hier werden Maßnahmen zur Qualitätssicherung und Qualitätsmerkmale festgelegt, die erfüllt sein müssen.
\end{document}