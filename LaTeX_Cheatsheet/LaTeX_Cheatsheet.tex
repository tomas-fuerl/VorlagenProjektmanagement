% Dieses Dokument dient als Cheatsheet für verschiedene Technologien in LaTeX, die ich benutze.
% Ich erhebe keinen Anspruch auf Vollständigkeit und Best Practices

% Deklaration der Variablen
\author{Tomas Fürl}
\newcommand{\Projektname}{Vorlagenprojekt}
\title{Anforderungen für \Projektname}
\documentclass[10pt,a4paper]{article}
\usepackage[utf8]{inputenc}
\usepackage[german]{babel}
\usepackage[german]{isodate}
\usepackage{tabularx}

% Ändere Paragraph zu Überschrift 4
\usepackage{titlesec}
\setcounter{secnumdepth}{4}
\titleformat{\paragraph}
{\normalfont\normalsize\bfseries}{\theparagraph}{1em}{}
\titlespacing*{\paragraph}
{0pt}{3.25ex plus 1ex minus .2ex}{1.5ex plus .2ex}

\usepackage[scaled]{helvet}
\renewcommand\familydefault{\sfdefault}

\begin{document}
% Titel, Inhaltsverzeichnis, Bildverzeichnis, Tabellenverzeichnis
\pagenumbering{gobble}
\maketitle
\newpage
\pagenumbering{Roman}
\tableofcontents
% \newpage
% \listoffigures
% \newpage
% \listoftables
\newpage
\pagenumbering{arabic}

% Liste
\section{Listenbeispiel}
    \begin{itemize}
        \item One entry in the list
        \item Another entry in the list
    \end{itemize}
    \begin{enumerate}
        \item One entry in the list
        \item Another entry in the list
    \end{enumerate}

% Tabelle
\section{Tabellenbeispiel}
    \begin{tabularx}{\textwidth}{|l|X|c|r|}
        \hline
        \textbf{Nr.} & \textbf{Beschreibung} & &\\
        \hline
        1. & Ich schreibe lange Texte, um meinen Standpunkt zu verdeutlicheStandpStandpunkt zu verdeutlichenunkt zu verdeutlichenn  & schrift  &  123 \\
        \hline
        2. & Ich schreibe lange Texte, um meinen Standpunkt zu verdeutlichen  & Schr  &   1234567 \\
        \hline
        3. & Ich schreibe lange Texte, um meinen Standpunkt   & Schriftift  &   12 \\
        \hline
        4. & Ich schreibe lange Texte, um meinen Standpunkt zu verdeutlichen  &   &   1234 \\
        \hline
    \end{tabularx}


\end{document}