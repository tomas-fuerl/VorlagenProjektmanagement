% Dieses Dokument ist eine Vorlage zum Erstellen einer Mindmap.

% Dokument hat Grundgröße für Schrift bei 12pt
\documentclass[12pt]{article}

% Deutsche Formate der Funktionen
\usepackage[ngerman]{babel}

% 
\usepackage{fullpage}

% Helvetica als Schriftart
\usepackage[scaled]{helvet}
\usepackage[T1]{fontenc}    
\usepackage{tikz}

% Library für Mindmaps nachladen
\usetikzlibrary{mindmap}

% Grafik auf Leere Seite plazieren
\pagestyle{empty}

% Beginn des Dokuments
\begin{document}

% Lade Grafik als Mindmap, Neue Objekte werden im Kreis abgebildet, Grundfarbe jedes Objektes ist Rot 50%
\begin{tikzpicture}[mindmap, grow cyclic, every node/.style=concept, concept color=red!50, 
    % 1. Ebene hat den Abstand 8cm und wird mit einem Winkel von 90°
    level 1/.append style={level distance=8cm,sibling angle=90},
    % 2. Ebene hat den Abstand 3cm und wird mit einem Winkel von 90°
    level 2/.append style={level distance=3cm,sibling angle=90},
    % 3. Ebene hat den Abstand 3cm und wird mit einem Winkel von 45°
    level 3/.append style={level distance=3cm,sibling angle=45}]
 
    \end{tikzpicture}
\end{document}